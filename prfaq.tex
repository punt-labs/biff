\documentclass[11pt,letterpaper]{article}

% ── Geometry ──────────────────────────────────────────────────────────────────
\usepackage[margin=1in]{geometry}

% ── Pagination ────────────────────────────────────────────────────────────────
\widowpenalty=10000          % No widow lines (last line alone at page top)
\clubpenalty=10000           % No orphan lines (first line alone at page bottom)
\brokenpenalty=10000         % No hyphenated words across page breaks
\raggedbottom                % Allow uneven page bottoms rather than stretching

% ── Fonts ─────────────────────────────────────────────────────────────────────
\usepackage[T1]{fontenc}
\usepackage{newpxtext}     % Palatino-based serif (TeX Gyre Pagella)
\usepackage{newpxmath}     % Matching math font

% ── Colors ────────────────────────────────────────────────────────────────────
\usepackage[dvipsnames]{xcolor}
\definecolor{SectionBlue}{HTML}{1B3A5C}
\definecolor{AccentGray}{HTML}{4A4A4A}
\definecolor{QuoteBorder}{HTML}{8B9DAF}
\definecolor{RiskRed}{HTML}{C0392B}
\definecolor{RiskAmber}{HTML}{E67E22}
\definecolor{RiskGreen}{HTML}{27AE60}
\definecolor{BoxBg}{HTML}{F7F8FA}

% ── Tables ────────────────────────────────────────────────────────────────────
\usepackage{booktabs}
\usepackage{tabularx}

% ── Boxes ─────────────────────────────────────────────────────────────────────
\usepackage[framemethod=tikz]{mdframed}

% ── Lists ─────────────────────────────────────────────────────────────────────
\usepackage{enumitem}

% ── Hyperlinks ────────────────────────────────────────────────────────────────
\usepackage{hyperref}
\hypersetup{
  colorlinks=true,
  linkcolor=SectionBlue,
  urlcolor=SectionBlue,
  pdfauthor={Jim Freeman},
  pdftitle={biff — PR/FAQ},
  pdfsubject={Working Backwards — CLI Communication for Engineers},
}

% ── Section styling ───────────────────────────────────────────────────────────
\usepackage{titlesec}
\titleformat{\section}
  {\Large\bfseries\color{SectionBlue}}
  {}
  {0pt}
  {}
  [\vspace{-0.5em}\textcolor{QuoteBorder}{\rule{\textwidth}{0.4pt}}]

\titleformat{\subsection}
  {\large\bfseries\color{AccentGray}}
  {}
  {0pt}
  {}

% ══════════════════════════════════════════════════════════════════════════════
% Custom environments
% ══════════════════════════════════════════════════════════════════════════════

% ── \prsection{title} — styled press release section header ──────────────────
\newcommand{\prsection}[1]{%
  \vspace{1em}%
  {\large\bfseries\color{SectionBlue}#1}%
  \vspace{0.3em}\par%
}

% ── customerquote — indented, italic customer quote ──────────────────────────
\newmdenv[
  topline=false,
  bottomline=false,
  rightline=false,
  linewidth=3pt,
  linecolor=QuoteBorder,
  backgroundcolor=BoxBg,
  innerleftmargin=12pt,
  innerrightmargin=12pt,
  innertopmargin=8pt,
  innerbottommargin=8pt,
  skipabove=\baselineskip,
  skipbelow=\baselineskip,
]{customerquote}

% ── spokespersonquote — indented spokesperson quote ──────────────────────────
\newmdenv[
  topline=false,
  bottomline=false,
  rightline=false,
  linewidth=3pt,
  linecolor=SectionBlue,
  backgroundcolor=BoxBg,
  innerleftmargin=12pt,
  innerrightmargin=12pt,
  innertopmargin=8pt,
  innerbottommargin=8pt,
  skipabove=\baselineskip,
  skipbelow=\baselineskip,
]{spokespersonquote}

% ── faqpair — Q&A pair with bold question ────────────────────────────────────
\newenvironment{faqpair}[1]{%
  \vspace{0.5em}%
  \noindent\textbf{\color{SectionBlue}Q: #1}\par%
  \vspace{0.2em}%
  \begin{adjustwidth}{1em}{0pt}%
}{%
  \end{adjustwidth}%
  \vspace{0.5em}%
}

% ── riskitem — description-style risk entry (avoids table margin issues) ─────
\newcommand{\riskitem}[2]{%
  \vspace{0.4em}%
  \noindent\textbf{\color{SectionBlue}#1}\par%
  \vspace{0.1em}%
  \begin{adjustwidth}{1em}{0pt}%
    #2%
  \end{adjustwidth}%
}

% ── adjustwidth (for faqpair indentation) ────────────────────────────────────
\usepackage{changepage}

% ══════════════════════════════════════════════════════════════════════════════
% Document
% ══════════════════════════════════════════════════════════════════════════════

\begin{document}

% ── Title block ───────────────────────────────────────────────────────────────
\begin{center}
  {\LARGE\bfseries\color{SectionBlue} biff} \\[0.5em]
  {\large\color{AccentGray} Team communication for engineers who never leave the terminal} \\[1.5em]
  {\small\color{AccentGray} February 2026 \quad|\quad Jim Freeman \quad|\quad punt-labs}
\end{center}

\vspace{1em}
\textcolor{QuoteBorder}{\rule{\textwidth}{0.8pt}}
\vspace{1em}

% ══════════════════════════════════════════════════════════════════════════════
% PRESS RELEASE
% ══════════════════════════════════════════════════════════════════════════════

\section*{Press Release}

\prsection{Summary}

San Francisco, CA --- Today, punt-labs announced \textbf{biff}, an open-source
communication tool that lets software engineers message teammates, share
context, and coordinate work without leaving their terminal. Unlike Slack and
Discord, which demand a separate window and constant attention, biff runs
inside the engineer's existing Claude Code session as MCP-native slash
commands. Engineers type \texttt{/mesg @kai} the same way they type
\texttt{git push} --- in flow, with intent, and then move on.

\prsection{Problem}

Software engineers using AI-assisted coding tools like Claude Code are
experiencing a step change in productivity. Entire features that once took
days now take hours. But every time these engineers need to coordinate with a
teammate --- asking a question, sharing a diff, requesting a code review ---
they context-switch to Slack or Discord, tools designed for managers and
all-day-online knowledge workers, not for makers in deep focus.

The cost is not hypothetical. Research consistently shows that a single
context switch costs 15--25 minutes of deep work recovery. For an engineer
shipping three features a day with AI assistance, three Slack interruptions
can erase an entire feature's worth of productive time. Slack's always-on
presence model compounds the problem: even \emph{checking} for messages
breaks flow, and the social expectation of quick replies creates anxiety that
degrades concentration even when no message arrives.

As Thomas Dohmke, former CEO of GitHub, observed when launching Entire in
February 2026: ``The terminal is becoming the new center of gravity, as
developers prompt fleets of agents across multiple terminal windows at
once.'' The engineers building the future of software --- the ones spending
their entire day inside a terminal, collaborating with AI agents, iterating
at speeds that would have seemed absurd two years ago --- have no
communication tool that matches how they actually work.

\prsection{Solution}

Biff brings communication into the terminal as slash commands borrowed from
the BSD Unix utilities that shipped with every workstation in the 1980s ---
\texttt{talk}, \texttt{wall}, \texttt{finger}, \texttt{who},
and \texttt{mesg} --- updated for the MCP era. A typical morning might look
like this:

\begin{quote}
\ttfamily
\small
/plan ``refactoring the auth layer'' \\
/mesg @kai ``auth module is ready for review'' \\
/cr @kai \\
/hive @kai @eric @jim \\
/wall ``I'm pulling the auth changes into the feature branch'' \\
\textnormal{\textit{...thirty minutes of focused work later...}} \\
/hive off
\end{quote}

Every command implies intent. There are no channels to monitor, no threads to
catch up on, no emoji reactions to parse. Communication is pull-based: when
someone sends you a message, your session notifies you, and you decide when
to read it. When you need deeper collaboration, \texttt{/pair @kai} invites a
teammate to send input to your Claude Code session --- and they can only join
when you explicitly consent. When a problem needs three people for thirty
minutes, \texttt{/hive} creates a temporary group that dissolves when the
work is done. The full command reference is in Appendix~A.

Because biff speaks MCP, it does not distinguish between a human session and
an agent session. An autonomous coding agent can join a \texttt{/hive},
announce progress via \texttt{/wall}, or \texttt{/mesg} a human when it needs
a decision. Agents can coordinate with other agents the same way. Biff is not
just communication for engineers --- it is the communication layer for the
entire hive of humans and agents building software together.

\prsection{Customer Quote}

\begin{customerquote}
\textit{``I used to mass-quit my Slack channels once a month out of
frustration and then guiltily rejoin them. With biff, I just set
\texttt{/biff off} when I'm deep in a problem and \texttt{/biff on} when I
come up for air. My teammates can \texttt{/finger @priya} to see what I'm on
without interrupting. Last week I shipped more in three days than I used to
ship in two weeks, and I didn't miss a single important message.''}

\raggedleft --- \textbf{Priya Chandrasekaran}, Senior Engineer at a Series B
startup
\end{customerquote}

\prsection{Getting Started}

Getting started with biff takes one command:
\texttt{pip install biff-mcp}. Biff registers itself as an MCP server in
your Claude Code session. Team configuration lives in a \texttt{.biff} file
committed to your git repository --- it contains the relay URL and
whitelisted member IDs. When you clone a repo that has a \texttt{.biff} file,
biff automatically connects you to the right team. There is no account to
create, no workspace to configure, and no browser to open. Type
\texttt{/who} to see your teammates. Within five minutes of installation, you
can send your first message, see your team's presence, and set your plan.

\prsection{Spokesperson Quote}

\begin{spokespersonquote}
``We built biff because we watched the best engineers in the world
disappear into their terminals --- shipping at speeds that make the old
sprint-and-standup cadence look quaint --- and then lose hours every day to
communication tools designed for a different era. Slack was built for the
open-office, always-online workplace. Biff is built for the deep-focus,
AI-accelerated one. We didn't add a chat feature to the terminal. We
resurrected the Unix communication vocabulary --- \texttt{talk},
\texttt{wall}, \texttt{finger}, \texttt{mesg} --- because those commands
understood something Slack forgot: communication should be purposeful, not
ambient.''

\raggedleft --- \textbf{Jim Freeman}, Founder, punt-labs
\end{spokespersonquote}

\prsection{Call to Action}

Biff is open source and available today at
\href{https://github.com/punt-labs/biff}{github.com/punt-labs/biff}.
Install with \texttt{pip install biff-mcp}. The core tool is free and always
will be. For teams that want hosted relay, presence guarantees, and audit
logs, biff teams will be available in Q3 2026.

\newpage

% ══════════════════════════════════════════════════════════════════════════════
% FREQUENTLY ASKED QUESTIONS
% ══════════════════════════════════════════════════════════════════════════════

\section*{Frequently Asked Questions}

% ── External FAQs (Customer-facing) ──────────────────────────────────────────

\subsection*{External FAQs}

\begin{faqpair}{What is biff and who is it for?}
  Biff is a terminal-native communication tool for software engineers who
  use Claude Code or other MCP-compatible systems. If you spend your working
  day in a terminal and resent switching to a browser to talk to your team,
  biff is for you. It provides eleven slash commands covering messaging,
  presence, collaboration, and code review --- all without leaving your
  session. See Appendix~A for the full command reference.
\end{faqpair}

\begin{faqpair}{How is biff different from Slack?}
  Slack is a workplace chat platform designed around channels, threads, and
  continuous presence. It assumes you are watching. Biff is a communication
  tool designed around intent and focus. It assumes you are working. In
  Slack, the default state is ``available and monitoring.'' In biff, the
  default state is ``heads down; interrupt me if it matters.'' Slack is
  optimized for managers who need visibility. Biff is optimized for
  engineers who need concentration.

  Structurally, biff runs inside your existing development environment as
  MCP slash commands. There is no separate app, no browser tab, no
  notification center. Communication happens where your code already lives.
\end{faqpair}

\begin{faqpair}{How is biff different from Discord?}
  Discord is a community platform built around persistent voice and text
  channels. It works well for open-source communities and gaming. Biff is a
  team tool built around directed messages and explicit intent. Discord is
  for hanging out. Biff is for getting things done. Biff also runs natively
  inside your AI coding session --- there is no alt-tab.
\end{faqpair}

\begin{faqpair}{How do I get started?}
  Run \texttt{pip install biff-mcp}. Biff auto-registers as an MCP server
  in your Claude Code session. If your repo has a \texttt{.biff} file
  (committed by a teammate), biff picks up the relay URL and team roster
  automatically. Type \texttt{/who} to see your team. No account creation,
  no workspace to configure. You are communicating within five minutes.
\end{faqpair}

\begin{faqpair}{What happens to my messages? Is communication private?}
  Messages are end-to-end encrypted in transit. In the open-source
  self-hosted mode, messages route through your own infrastructure and
  nothing is stored by punt-labs. In the hosted relay mode (biff teams),
  messages are encrypted at rest and subject to a retention policy you
  configure. Biff never reads, analyzes, or trains on your messages.
\end{faqpair}

\begin{faqpair}{Does biff work without Claude Code?}
  Biff is built on the Model Context Protocol (MCP), which is an open
  standard. Any MCP-compatible client can use biff. Claude Code is the
  primary target today because it has the largest population of engineers
  living in the terminal. As other MCP clients mature, biff will work there
  too.
\end{faqpair}

\begin{faqpair}{What does ``pairing'' mean? Can someone control my session?}
  When a teammate invokes \texttt{/pair @you}, they are requesting to send
  input to your Claude Code session --- not the other way around. You see
  the request and explicitly accept or decline. If you accept, the
  teammate can type prompts that are sent to your Claude, but you retain
  full control and can see everything they send. You can end the pairing
  at any time. No one can read your session, execute commands, or modify
  your files without your explicit, per-session consent.

  This is different from \texttt{/talk}, which is a simple real-time text
  conversation between two terminals --- no session access involved.
\end{faqpair}

\begin{faqpair}{Can agents use biff?}
  Yes. Because biff is an MCP server, any MCP-compatible agent can use it
  the same way a human does. An autonomous coding agent can join a
  \texttt{/hive}, broadcast status via \texttt{/wall}, send artifacts via
  \texttt{/send}, or message a human when it needs a decision. Agents can
  also communicate with other agents --- coordinating work, handing off
  tasks, or requesting reviews. Biff does not distinguish between human
  and agent sessions. If it speaks MCP, it can participate.
\end{faqpair}

% ── Internal FAQs (Business-facing) ──────────────────────────────────────────

\subsection*{Internal FAQs}

\subsubsection*{Value \& Market}

\begin{faqpair}{What is the total addressable market?}
  The immediate market is engineers using AI coding assistants inside
  terminals. GitHub reports over 100 million developers on the platform.
  AI-assisted coding adoption is growing rapidly --- GitHub Copilot alone
  surpassed 1.8 million paid subscribers by late 2024, and Claude Code
  launched to strong adoption in early 2025. The subset of engineers who
  work primarily in the terminal with MCP-compatible tools is small today
  (estimated tens of thousands) but growing fast as MCP becomes the
  standard protocol for AI tool integration.

  Bottoms-up: if 50,000 engineers adopt biff in year one, and 20\% of them
  are on teams that convert to the paid hosted relay at \$10/user/month,
  that represents \$1.2M ARR. The real opportunity scales in two
  dimensions: as MCP adoption grows, the human addressable market
  multiplies; and as autonomous agents become standard members of
  engineering teams, the number of sessions that need coordination grows
  faster than headcount.
\end{faqpair}

\begin{faqpair}{What evidence do we have that customers want this?}
  Four categories of evidence:

  \begin{enumerate}[leftmargin=1.5em]
    \item \textbf{Behavioral signal:} The explosive growth of CLI-native
    tooling (Claude Code, Cursor's terminal mode, Warp, Ghostty) demonstrates
    that engineers are consolidating their workflow into the terminal. The
    ecosystem of MCP servers, Claude Code plugins, and slash commands exists
    precisely because engineers want to \emph{stay in flow}. Communication
    is the last major activity that forces them out.

    \item \textbf{Cultural signal:} Steve Yegge's February 2026 article
    ``The Anthropic Hive
    Mind''\footnote{\url{https://steve-yegge.medium.com/the-anthropic-hive-mind-d01f768f3d7b}}
    describes Anthropic's internal engineering culture where teams ``swarm
    around campfires'' and ``the whole team is pair programming at once''
    --- all happening inside Claude Code sessions. He reports engineers
    achieving 10--100x productivity gains, but notes that ``you need full
    transparency at all times, at their speeds, or nobody will ever see
    what you are doing.'' This is exactly the problem biff solves:
    visibility and coordination at AI speed, inside the tool.

    \item \textbf{Investment signal:} Thomas Dohmke, former CEO of GitHub,
    launched Entire in February 2026 with a \$60M seed round --- the largest
    seed investment ever for a developer tools
    startup.\footnote{\url{https://entire.io/blog/hello-entire-world/}}
    His thesis: ``the entire software ecosystem is being bottlenecked by a
    manual system of production'' and ``the terminal is becoming the new
    center of gravity.'' Entire focuses on the code context layer (how AI
    changes get reviewed and governed). Biff focuses on the human
    coordination layer (how engineers communicate while working at AI
    speed). The same tectonic shift creates both opportunities.

    \item \textbf{Historical precedent:} The original BSD Unix included
    \texttt{write}, \texttt{talk}, \texttt{wall}, \texttt{finger}, and
    \texttt{mesg} as standard utilities because the designers understood
    that engineers who share a system need lightweight, intentional
    communication. Those tools died when engineers left the terminal for
    GUIs. Now that engineers are returning to the terminal, the need has
    returned with them.
  \end{enumerate}
\end{faqpair}

\begin{faqpair}{Who are the competitors and why will we win?}
  \textbf{Slack} is the dominant team communication tool but is
  increasingly resented by engineers for its interruption-driven model.
  Slack could build a CLI client, but its revenue depends on engagement
  metrics (messages sent, channels monitored, time-in-app) that are
  fundamentally opposed to biff's focus-first model. Slack's incentives
  make it structurally unable to optimize for fewer interruptions.

  \textbf{Discord} serves developer communities well but is not a team
  coordination tool. Its voice-channel model does not map to directed,
  intentional engineering communication.

  \textbf{GitHub} has discussions, issues, and PR comments --- all
  web-based, all requiring a browser. The \texttt{gh} CLI covers some
  workflows but offers no real-time messaging or presence.

  \textbf{Zulip} and \textbf{Mattermost} are open-source Slack
  alternatives with the same fundamental model: channels, threads, browser
  or desktop app. They solve the vendor-lock-in problem but not the
  context-switch problem.

  Biff wins by refusing to play Slack's game. It is not a chat app with a
  CLI client. It is a communication protocol native to the environment
  where engineers already work. The moat is the MCP ecosystem: as more
  tools adopt MCP, biff's integration surface grows without additional
  development.
\end{faqpair}

\subsubsection*{Technical}

\begin{faqpair}{What are the major technical risks?}
  \begin{enumerate}[leftmargin=1.5em]
    \item \textbf{MCP protocol maturity (medium risk).} MCP is young. The
    spec may change in ways that require rearchitecture. Mitigation: biff
    wraps MCP interactions behind an internal abstraction layer, and
    punt-labs actively tracks the MCP spec evolution.

    \item \textbf{Message relay infrastructure (low risk).} Routing
    messages between engineers on different machines requires a relay
    server. This is well-understood infrastructure (WebSocket/SSE relay).
    The self-hosted option reduces dependency on punt-labs infrastructure.

    \item \textbf{Security of pairing (medium risk).} The \texttt{/pair}
    feature, which allows one engineer to send input to another's Claude
    Code session, requires careful permission design. Mitigation: explicit
    per-session consent, all input visible to the session owner, all
    interactions logged, session owner retains kill switch.

    \item \textbf{Adoption chicken-and-egg (high risk).} Biff is only
    useful if your teammates also use it. Mitigation: the \texttt{.biff}
    file committed to the repo means the second person on a team only
    needs \texttt{pip install biff-mcp} --- the configuration is already
    there. The \texttt{/plan}, \texttt{/finger}, and \texttt{/who}
    commands provide value to individual users even without teammates.
    The \texttt{/send} and \texttt{/cr} commands provide enough value in
    pairs to bootstrap adoption within a team.
  \end{enumerate}
\end{faqpair}

\begin{faqpair}{What dependencies exist on other teams or systems?}
  Biff depends on the MCP protocol (maintained by Anthropic), Claude Code
  as the primary host client (also Anthropic), and Python packaging
  infrastructure (PyPI). There are no dependencies on other punt-labs
  products. The relay server is self-contained and can be self-hosted.
\end{faqpair}

\begin{faqpair}{What is the estimated development timeline?}
  \begin{description}[leftmargin=1.5em, labelwidth=6em, font=\bfseries]
    \item[Phase 1 (4 weeks)] Core MCP server with \texttt{/mesg},
    \texttt{/who}, \texttt{/plan}, \texttt{/finger}, \texttt{/biff on/off}.
    Local relay for same-machine communication. Single-user value from
    \texttt{/plan} and session awareness.
    \item[Phase 2 (4 weeks)] Network relay for cross-machine messaging.
    \texttt{/wall}, \texttt{/hive}, \texttt{/send}, \texttt{/cr}. Team
    communication functional end-to-end.
    \item[Phase 3 (4 weeks)] \texttt{/talk} real-time conversation,
    \texttt{/pair} with permission model. End-to-end encryption. Security
    audit.
    \item[Phase 4 (4 weeks)] Hosted relay service (biff teams).
    Admin controls, audit logs, retention policies. Billing integration.
  \end{description}

  Total: approximately 16 weeks to full product with hosted offering. Phase
  1 delivers usable value and can be released as a public beta.
\end{faqpair}

\begin{faqpair}{What is the scaling story?}
  The relay server is a stateless WebSocket router. Horizontal scaling is
  straightforward --- add more relay nodes behind a load balancer. The
  bottleneck at scale is presence: broadcasting \texttt{/who} status across
  thousands of users requires a gossip protocol or presence service. This
  is a solved problem (XMPP, Matrix) and is not needed until biff reaches
  thousands of concurrent users.
\end{faqpair}

\subsubsection*{Business}

\begin{faqpair}{What is the revenue model?}
  Open core. The biff CLI tool and self-hosted relay are free and open
  source (MIT license). Revenue comes from \textbf{biff teams}, a hosted
  relay service priced at \$10/user/month, offering:

  \begin{itemize}[leftmargin=1.5em]
    \item Managed relay with 99.9\% uptime SLA
    \item End-to-end encryption with key management
    \item Audit logs and compliance exports
    \item Team admin controls (permissions, onboarding, offboarding)
    \item Priority support
  \end{itemize}

  This model is proven by GitLab, Sentry, PostHog, and other
  developer-tools companies. Engineers adopt the free tool; their companies
  pay for the managed version when the team exceeds 5--10 people and needs
  reliability guarantees.
\end{faqpair}

\begin{faqpair}{What are the key metrics for success?}
  \begin{description}[leftmargin=1.5em, labelwidth=10em, font=\bfseries]
    \item[Weekly active users] Target: 1,000 WAU within 6 months of
    launch. Measures adoption.
    \item[Messages per user/week] Target: 15+ messages/week for active
    users. Measures engagement without rewarding noise --- biff's
    intentional model should show \emph{fewer} messages per user than
    Slack but higher signal-to-noise.
    \item[Team conversion rate] Target: 10\% of teams with 3+ active free
    users convert to biff teams within 90 days.
    \item[Retention (D30)] Target: 40\% of installers still active after
    30 days. CLI tools typically see high install-and-forget rates; 40\%
    retention indicates genuine workflow integration.
  \end{description}

  Decision threshold: if WAU has not reached 500 within 4 months of
  public launch, revisit the go-to-market strategy. If D30 retention is
  below 20\%, the product is not sticky enough and needs fundamental
  rethinking.
\end{faqpair}

\begin{faqpair}{Why now? What has changed?}
  Three things changed simultaneously, and the convergence is visible in
  real dollars and real behavior:

  \begin{enumerate}[leftmargin=1.5em]
    \item \textbf{Engineers returned to the terminal.} After two decades of
    IDE dominance, AI coding tools brought engineers back to the command
    line. Claude Code, Cursor's terminal mode, and the MCP ecosystem mean
    the terminal is once again the primary workspace for a growing
    population of engineers. Dohmke calls this ``the new center of
    gravity'' and bet \$60M on it. Yegge describes Anthropic engineers who
    ``never leave their terminal sessions.''

    \item \textbf{MCP created a standard protocol.} Before MCP, building a
    communication tool inside a coding environment meant integrating with
    every editor separately. MCP provides a single integration point that
    works across any compatible client. Biff integrates once and works
    everywhere MCP does.

    \item \textbf{AI-accelerated engineers outgrew Slack.} When engineers
    ship 3--5x faster, the communication overhead that was tolerable at the
    old pace becomes the bottleneck. Slack's interruption model was
    designed for a world where shipping a feature took a sprint. In a world
    where shipping a feature takes a morning, Slack's model actively
    harms productivity.
  \end{enumerate}
\end{faqpair}

\begin{faqpair}{What are we not building?}
  Biff is not building:
  \begin{itemize}[leftmargin=1.5em]
    \item \textbf{Persistent channels.} Biff has \texttt{/hive} for
    temporary groups that dissolve when the work is done, not channels that
    accumulate forever. There is no concept of a persistent room.
    \item \textbf{Message history or search.} Messages are ephemeral by
    default. If you need a record, use \texttt{/send} to send an artifact
    to a durable store (git, GitHub Issues). Biff is not a system of
    record.
    \item \textbf{Video or voice.} Biff is text. If you need a video call,
    use the tool built for video calls.
    \item \textbf{Project management.} No tasks, no boards, no sprints.
    Biff is communication, not coordination. Use beads, Linear, or GitHub
    Issues for project tracking.
    \item \textbf{A mobile app.} Biff is for engineers at their terminals.
    If you are on your phone, you are not in the target workflow.
  \end{itemize}
\end{faqpair}

% ══════════════════════════════════════════════════════════════════════════════
% FOUR RISKS ASSESSMENT
% ══════════════════════════════════════════════════════════════════════════════

\section*{Risk Assessment}

\begin{mdframed}[
  linewidth=0.5pt,
  linecolor=QuoteBorder,
  backgroundcolor=BoxBg,
  innerleftmargin=10pt,
  innerrightmargin=10pt,
  innertopmargin=10pt,
  innerbottommargin=10pt,
]
{\bfseries\color{SectionBlue}Four Risks Assessment}\par
\vspace{0.5em}

\riskitem{Value}{%
  \textbf{Medium risk.} Strong circumstantial evidence (CLI adoption trend,
  MCP ecosystem growth, Yegge's observations about AI-speed collaboration)
  but no direct customer validation yet. The risk is not ``do engineers
  dislike Slack?'' (they do) but ``is the pain severe enough to adopt a new
  tool?'' Slack's network effects are powerful --- even engineers who hate
  Slack use it because their team does. Biff must provide enough standalone
  value (\texttt{/plan}, \texttt{/finger}) to bootstrap adoption before
  network effects kick in. The artisanal approach --- building for ourselves
  first --- provides an honest signal: if we do not use it daily, no one
  else will either.}

\riskitem{Usability}{%
  \textbf{Low risk.} The slash-command interface mirrors patterns engineers
  already use (git commands, Unix utilities, Claude Code slash commands).
  Installation is a single \texttt{pip install}. Time-to-value is under five
  minutes. The primary usability risk is discoverability: new users need to
  learn the command vocabulary. Mitigation: the vocabulary is small (11
  commands), the names are intuitive (borrowed from Unix), and
  \texttt{/help} documents everything inline.}

\riskitem{Feasibility}{%
  \textbf{Low-to-medium risk.} The core technology --- MCP servers,
  WebSocket relay, message routing --- is well-understood. The team has
  direct experience with MCP server development and Python CLI tooling.
  The highest-risk component is the \texttt{/pair} feature,
  which requires careful permission design for session input sharing. This is scoped to Phase 3,
  giving time for the security model to mature. No component requires
  novel technology.}

\riskitem{Viability}{%
  \textbf{Medium risk.} The open-core model is proven in developer tools,
  but biff's intentional low-volume communication model means engagement
  metrics will look different from typical SaaS. Investors and boards
  accustomed to Slack-style engagement numbers may question whether ``15
  messages per user per week'' constitutes a healthy product. The counter:
  biff's value is measured in time saved, not messages sent. The hosted
  relay's value proposition (uptime, encryption, compliance) is clear for
  teams above 5--10 people. The risk is reaching that critical mass of
  team adoption quickly enough to demonstrate the conversion funnel.}

\end{mdframed}

\newpage

% ══════════════════════════════════════════════════════════════════════════════
% APPENDIX A: COMMAND REFERENCE
% ══════════════════════════════════════════════════════════════════════════════

\section*{Appendix A: Command Reference}

\renewcommand{\arraystretch}{1.4}
\begin{tabularx}{\textwidth}{@{}l l X@{}}
  \toprule
  \textbf{Command} & \textbf{Origin} & \textbf{Description} \\
  \midrule

  \texttt{/mesg @user ``text''} & BSD \texttt{mesg} &
  Send a one-way message to a user. Asynchronous --- the recipient reads
  it when they choose. \\

  \texttt{/talk @user} & BSD \texttt{talk} &
  Open a real-time bidirectional conversation. Both users type and read
  in their own terminals. Either side can end the session. \\

  \texttt{/wall ``text''} & BSD \texttt{wall} &
  Broadcast a message. If you are in a \texttt{/hive}, the message goes
  to hive members. Otherwise it goes to your full team. \\

  \texttt{/finger @user} & BSD \texttt{finger} &
  Read a user's plan (set via \texttt{/plan}) and current status.
  Non-intrusive --- the target is not notified. \\

  \texttt{/who} & BSD \texttt{who} &
  List all active sessions. Shows who is online, their availability
  (\texttt{/biff on/off}), and their current plan. \\

  \texttt{/plan ``text''} & BSD \texttt{.plan} &
  Set your status message. Visible to anyone who runs \texttt{/finger}
  or \texttt{/who}. Update it as your work changes. \\

  \texttt{/biff on} | \texttt{off} & BSD \texttt{biff} &
  Control whether you receive messages. When off, incoming messages are
  queued until you turn it back on. \\

  \texttt{/hive @a @b @c} & --- &
  Create a temporary group with the named users. All members see each
  other's messages. Use \texttt{/wall} to broadcast within the hive.
  \texttt{/hive off} dissolves the group. \\

  \texttt{/pair @user} & --- &
  Invite a user to send input to your Claude Code session. Requires
  explicit acceptance. The session owner sees all input and retains full
  control. \\

  \texttt{/send @user} & --- &
  Send a diff, file, or code snippet to a user with full context.
  The recipient receives the artifact inline in their session. \\

  \texttt{/cr @user} & --- &
  Request a code review. Automatically attaches the relevant changes
  from your current branch. The reviewer receives the diff and can
  respond with comments. \\

  \bottomrule
\end{tabularx}

\end{document}
